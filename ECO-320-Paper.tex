% Options for packages loaded elsewhere
\PassOptionsToPackage{unicode}{hyperref}
\PassOptionsToPackage{hyphens}{url}
\PassOptionsToPackage{dvipsnames,svgnames,x11names}{xcolor}
%
\documentclass[
  12pt,
  letterpaper,
  DIV=11,
  numbers=noendperiod]{scrartcl}

\usepackage{amsmath,amssymb}
\usepackage{setspace}
\usepackage{iftex}
\ifPDFTeX
  \usepackage[T1]{fontenc}
  \usepackage[utf8]{inputenc}
  \usepackage{textcomp} % provide euro and other symbols
\else % if luatex or xetex
  \usepackage{unicode-math}
  \defaultfontfeatures{Scale=MatchLowercase}
  \defaultfontfeatures[\rmfamily]{Ligatures=TeX,Scale=1}
\fi
\usepackage{lmodern}
\ifPDFTeX\else  
    % xetex/luatex font selection
\fi
% Use upquote if available, for straight quotes in verbatim environments
\IfFileExists{upquote.sty}{\usepackage{upquote}}{}
\IfFileExists{microtype.sty}{% use microtype if available
  \usepackage[]{microtype}
  \UseMicrotypeSet[protrusion]{basicmath} % disable protrusion for tt fonts
}{}
\usepackage{xcolor}
\setlength{\emergencystretch}{3em} % prevent overfull lines
\setcounter{secnumdepth}{5}
% Make \paragraph and \subparagraph free-standing
\ifx\paragraph\undefined\else
  \let\oldparagraph\paragraph
  \renewcommand{\paragraph}[1]{\oldparagraph{#1}\mbox{}}
\fi
\ifx\subparagraph\undefined\else
  \let\oldsubparagraph\subparagraph
  \renewcommand{\subparagraph}[1]{\oldsubparagraph{#1}\mbox{}}
\fi


\providecommand{\tightlist}{%
  \setlength{\itemsep}{0pt}\setlength{\parskip}{0pt}}\usepackage{longtable,booktabs,array}
\usepackage{calc} % for calculating minipage widths
% Correct order of tables after \paragraph or \subparagraph
\usepackage{etoolbox}
\makeatletter
\patchcmd\longtable{\par}{\if@noskipsec\mbox{}\fi\par}{}{}
\makeatother
% Allow footnotes in longtable head/foot
\IfFileExists{footnotehyper.sty}{\usepackage{footnotehyper}}{\usepackage{footnote}}
\makesavenoteenv{longtable}
\usepackage{graphicx}
\makeatletter
\def\maxwidth{\ifdim\Gin@nat@width>\linewidth\linewidth\else\Gin@nat@width\fi}
\def\maxheight{\ifdim\Gin@nat@height>\textheight\textheight\else\Gin@nat@height\fi}
\makeatother
% Scale images if necessary, so that they will not overflow the page
% margins by default, and it is still possible to overwrite the defaults
% using explicit options in \includegraphics[width, height, ...]{}
\setkeys{Gin}{width=\maxwidth,height=\maxheight,keepaspectratio}
% Set default figure placement to htbp
\makeatletter
\def\fps@figure{htbp}
\makeatother
\newlength{\cslhangindent}
\setlength{\cslhangindent}{1.5em}
\newlength{\csllabelwidth}
\setlength{\csllabelwidth}{3em}
\newlength{\cslentryspacingunit} % times entry-spacing
\setlength{\cslentryspacingunit}{\parskip}
\newenvironment{CSLReferences}[2] % #1 hanging-ident, #2 entry spacing
 {% don't indent paragraphs
  \setlength{\parindent}{0pt}
  % turn on hanging indent if param 1 is 1
  \ifodd #1
  \let\oldpar\par
  \def\par{\hangindent=\cslhangindent\oldpar}
  \fi
  % set entry spacing
  \setlength{\parskip}{#2\cslentryspacingunit}
 }%
 {}
\usepackage{calc}
\newcommand{\CSLBlock}[1]{#1\hfill\break}
\newcommand{\CSLLeftMargin}[1]{\parbox[t]{\csllabelwidth}{#1}}
\newcommand{\CSLRightInline}[1]{\parbox[t]{\linewidth - \csllabelwidth}{#1}\break}
\newcommand{\CSLIndent}[1]{\hspace{\cslhangindent}#1}

\KOMAoption{captions}{tableheading}
\usepackage{indentfirst}
\usepackage{setspace}
\makeatletter
\makeatother
\makeatletter
\makeatother
\makeatletter
\@ifpackageloaded{caption}{}{\usepackage{caption}}
\AtBeginDocument{%
\ifdefined\contentsname
  \renewcommand*\contentsname{Table of contents}
\else
  \newcommand\contentsname{Table of contents}
\fi
\ifdefined\listfigurename
  \renewcommand*\listfigurename{List of Figures}
\else
  \newcommand\listfigurename{List of Figures}
\fi
\ifdefined\listtablename
  \renewcommand*\listtablename{List of Tables}
\else
  \newcommand\listtablename{List of Tables}
\fi
\ifdefined\figurename
  \renewcommand*\figurename{Figure}
\else
  \newcommand\figurename{Figure}
\fi
\ifdefined\tablename
  \renewcommand*\tablename{Table}
\else
  \newcommand\tablename{Table}
\fi
}
\@ifpackageloaded{float}{}{\usepackage{float}}
\floatstyle{ruled}
\@ifundefined{c@chapter}{\newfloat{codelisting}{h}{lop}}{\newfloat{codelisting}{h}{lop}[chapter]}
\floatname{codelisting}{Listing}
\newcommand*\listoflistings{\listof{codelisting}{List of Listings}}
\makeatother
\makeatletter
\@ifpackageloaded{caption}{}{\usepackage{caption}}
\@ifpackageloaded{subcaption}{}{\usepackage{subcaption}}
\makeatother
\makeatletter
\@ifpackageloaded{tcolorbox}{}{\usepackage[skins,breakable]{tcolorbox}}
\makeatother
\makeatletter
\@ifundefined{shadecolor}{\definecolor{shadecolor}{rgb}{.97, .97, .97}}
\makeatother
\makeatletter
\makeatother
\makeatletter
\makeatother
\ifLuaTeX
  \usepackage{selnolig}  % disable illegal ligatures
\fi
\IfFileExists{bookmark.sty}{\usepackage{bookmark}}{\usepackage{hyperref}}
\IfFileExists{xurl.sty}{\usepackage{xurl}}{} % add URL line breaks if available
\urlstyle{same} % disable monospaced font for URLs
\hypersetup{
  pdftitle={Paper},
  pdfauthor={Diana Liu},
  colorlinks=true,
  linkcolor={blue},
  filecolor={Maroon},
  citecolor={Blue},
  urlcolor={Blue},
  pdfcreator={LaTeX via pandoc}}

\title{Paper}
\author{Diana Liu}
\date{2024-03-13}

\begin{document}
\maketitle
\begin{abstract}
uhh abstract i gues..
\end{abstract}
\ifdefined\Shaded\renewenvironment{Shaded}{\begin{tcolorbox}[breakable, boxrule=0pt, frame hidden, interior hidden, borderline west={3pt}{0pt}{shadecolor}, enhanced, sharp corners]}{\end{tcolorbox}}\fi

\setstretch{2}
\hypertarget{introduction}{%
\section{Introduction}\label{introduction}}

This paper will examine the efficiency of legally requiring publicly
traded companies to disclose certain financial and operational
information by applying economic theories developed by Kronman (1978)
and Cooter and Ulen (2016). This paper was written and compiled in R
Core Team (2023).

\hypertarget{required-disclosure-of-asymmetric-information-in-the-status-quo}{%
\section{Required Disclosure of Asymmetric Information in the Status
Quo}\label{required-disclosure-of-asymmetric-information-in-the-status-quo}}

Public companies are companies that have their stocks traded on an
exchange like the New York Stock Exchange.The biggest benefit to being
publicly traded is to raise funds from investors purchasing your stock.
However, in order to be publicly traded, companies are required by law
in Canada and the U.S. to regularly make information about their
activities and financial status available to the public (Commission
2024). This process is known as continuous disclosure and is achieved
through publishing quarterly and annual reports, like financial
statements, executive compensation, and forward looking information
circulars (Commission 2024). Continuous disclosure provides valuable
information that investors and other stakeholders like unions,
customers, and banks use to make decisions about the company in
question. This information requires deliberate investment by the company
to collect and organize. Costs like employing accountants and investing
in internal record keeping systems to sum up and compile thousands or
potentially hundreds of thousands of transactions in any given year are
expensive. Often companies already have internal systems to acquire
information for business decision making, resulting in asymmetric
information between companies and their stakeholders. Required
disclosure of this information in the status quo turns this information
from the property of the company into a public good, solving this
asymmetry, but incurs costs and benefits of its own.

\hypertarget{stakeholders}{%
\subsection{Stakeholders}\label{stakeholders}}

A stakeholder in a company is an individual or organization whose well
being will be effected by the financial and operational health of the
company. This means that they directly benefit from required disclosure
where financial and operational information that they want to know are
shared at no cost to them. For example, creditors who are owed money by
the company want to know if the company has enough income to repay them.
Other stakeholders include shareholders, suppliers, customers,
employees, and unions among others.

\hypertarget{efficiency-of-required-disclosure}{%
\section{Efficiency of Required
Disclosure}\label{efficiency-of-required-disclosure}}

\hypertarget{cost-savings}{%
\subsection{Cost Savings}\label{cost-savings}}

If it is assumed that it costs the same amount for a stakeholder to
gather continuous disclosure information as the company itself, then
requiring disclosure saves for stakeholders the socially wasteful costs
of searching for information that the company already has (Eisenberg
2003). To illustrate this, imagine that company A is negotiating a loan
contract with a bank and they are not required by law to disclose any
information. The bank want to know the financial information of company
A so that they can be confident that their loan will be repaid with
interest in the future. It is less costly for company A to acquire the
relevant information than the bank because similar information is used
for internal record keeping. If company A values the loan more than the
costs of acquiring the information, then they will share the information
with the bank, especially if they are financially successful and believe
that the information increases the chances of securing a loan. This
saves the bank costs of acquiring the information. On the other hand, if
company A is not financially successful and believe that the information
will decrease their chances of getting a loan, they will not willingly
share the information, forcing the bank to spend money gathering
information that company A already has, creating a social inefficiency.
In this case, compelling the company to publish this information by law
saves the bank the cost of searching for information.

\hypertarget{time-savings}{%
\subsection{Time Savings}\label{time-savings}}

Outside of information gathering costs, the immediate availability of
continuous disclosure information also allows faster allocation of goods
to their highest-valued uses. Suppose that company B is competing for
the same loan contract and has higher value for the loan than company A.
If the bank takes time to search for company A's financial information,
the loan will remain unallocated until the bank gathers enough
information to decide which company to loan to. This is less efficient
that if both companies' information were readily available because the
bank can find the most efficient allocation of the loan immediately,
saving the time it takes to transfer the loan to its highest value user.

\hypertarget{correcting-mistaken-assumptions}{%
\subsection{Correcting Mistaken
Assumptions}\label{correcting-mistaken-assumptions}}

Previous examples have been instances where parties are negotiating a
one-time contract, but companies are usually engaged with routine or
continuous transactions with stakeholders. Kronman (1978) demonstrates
that ``allocative efficiency is promoted by getting information of
changed circumstances to the market as quickly as possible so that
parties affected by the information can alter their behaviour to waste
as little resources as possible''. This is especially important for
continuous disclosure where information is published on a quarterly
basis.

Suppose that companies are only required to disclose their financial and
operational information once and not continuously. Employees are engaged
by a company through employment contracts. At the negotiation of the
contract, the employees taken into account the disclosed information and
believe the company is financially healthy and is able to pay their
salaries in the future. Some time passes and the company is no longer
financially healthy and will go bankrupt but they are not required to
disclose this fact. The employee's assumptions are now mistaken. In
order for them to correct their mistaken assumption, they must gather
their company's financial information periodically which is a huge cost
to each employee, likely outweighing the benefits. In this scenario, it
is most efficient for the employees to leave their company for new jobs
that are able to continue to pay them, but they do not due to their
mistaken assumption.

\hypertarget{cost-benefit-trade-offs}{%
\subsection{Cost Benefit Trade-Offs}\label{cost-benefit-trade-offs}}

\hypertarget{overreliance}{%
\subsection{Overreliance}\label{overreliance}}

\newpage

\hypertarget{references}{%
\section*{References}\label{references}}
\addcontentsline{toc}{section}{References}

\hypertarget{refs}{}
\begin{CSLReferences}{1}{0}
\leavevmode\vadjust pre{\hypertarget{ref-CiteOSC}{}}%
Commission, Ontario Securities. 2024. \emph{Continuous Disclosure}.
\href{https://www.osc.ca/en/industry/companies/continuous-disclosure\#:~:text=Companies\%20that\%20are\%20reporting\%20issuers}{www.osc.ca/en/industry/companies/continuous-disclosure\#:\textasciitilde:text=Companies\%20that\%20are\%20reporting\%20issuers}.

\leavevmode\vadjust pre{\hypertarget{ref-CiteCooter}{}}%
Cooter, Robert, and Thomas Ulen. 2016. \emph{Law and Economics}. 6th ed.
Berkeley, California: Berkeley Law Books.
\url{http://www.econ.jku.at/t3/staff/winterebmer/teaching/law_economics/ss19/6th_edition.pdf}.

\leavevmode\vadjust pre{\hypertarget{ref-CiteEisenberg}{}}%
Eisenberg, Melvin A. 2003. {``Disclosure in Contract Law.''}
\emph{California Law Review} 91 (6).
\url{https://doi.org/10.2307/3481400}.

\leavevmode\vadjust pre{\hypertarget{ref-CiteKronman}{}}%
Kronman, Anthony T. 1978. {``Mistake, Disclosure, Information, and the
Law of Contracts.''} \emph{The Journal of Legal Studies} 7 (1).
\url{https://doi.org/10.1086/467583}.

\leavevmode\vadjust pre{\hypertarget{ref-CiteR}{}}%
R Core Team. 2023. \emph{R: A Language and Environment for Statistical
Computing}. Vienna, Austria: R Foundation for Statistical Computing.
\url{https://www.R-project.org/}.

\end{CSLReferences}



\end{document}
